\documentclass{beamer}

% package hyperref needs these to be set here

\author{Pritvi Jheengut @zcoldplayer}
\subject{DevConMU - Protocols for Communication}

\usepackage[size=a4, orientation=landscape, scale=3.5]{beamerposter}

\usepackage{lmodern}
\usepackage{fontenc}[T1]
\usepackage{inputenc}[utf8]
\usepackage{hyperref}[pdfpagelabels=true,bookmarks=true, unicode=true]
\usepackage{pgf}
%\usepackage{verbatim}

\usepackage{listings}
\usepackage{xy}[all]%% License to be introduced


%  Eliminate errors such as
%  LaTeX Font Warning: Font shape `T1/cmss/m/n' in size <4> not available
%  LaTeX Font Warning: Size substitutions with differences up to 1.0pt 

\mode<presentation>{
  
  \useoutertheme{infolines}
  \usecolortheme{whale}
  
  \definecolor{ncurses}{HTML}{B7DFFF}
  \definecolor{darkblue}{HTML}{253789}
  \definecolor{light}{HTML}{A2CDEE}
  \definecolor{confgreen}{HTML}{3CA387}
  
  \setbeamercolor{Title bar}{fg=black!50}
  \setbeamercolor{frametitle}{parent=Title bar}
  \setbeamercolor{titlelike}{parent=light,bg=confgreen,fg=white}
  \setbeamercolor{block title}{bg=darkblue,fg=light}
  
  \setbeamercovered{transparent,dynamic}
  
  \setbeamertemplate{blocks}[rounded][shadow=true]
  \setbeamertemplate{background canvas}[vertical shading][top=
    confgreen!35,middle=light!5,bottom=red!50]
                    
  \mode<handout>{\beamertemplatesolidbackgroundcolor{black!50}}
                    
}

\begin{document}

\section{The Makers}

\title[DevConMU - Protocols for Communication]{Protocols for
  Communication by Makers}
\subtitle{Overview of Protocols to make your devices and gadgets talk
  to each other or/and their shields.}

\date[Developers Conference - Accelerator]{Developers Conference -
  18 May 2018}

\maketitle

\subsection{CopyLeft License}

\frame{
  \frametitle{Copyleft License Attribution}
  
  Made with love using beamer, LaTeX and git.
  
  \begin{alertblock}{This work is licensed under the LaTeX Project
      Public License.}
    To view a copy of this license, visit \\
    \url{https://www.latex-project.org/lppl.txt}    
  \end{alertblock}
  
  \begin{alertblock}{This work is licensed under the Creative
      Commons Attribution 4.0 International License.}
    To view a copy of this license, visit \\
    \url{http://creativecommons.org/licenses/by/4.0/} or \\
    
    send a letter to \\
    Creative Commons, \\
    PO Box 1866, \\
    Mountain View, \\
    CA 94042, \\
    USA. \\
  \end{alertblock}
}

\subsection{The  Mauritius Makers Community}

\frame{
  \frametitle{An introduction to the  Mauritius Makers Community -
    Part 1}
  
  \begin{block}{The Mauritius Makers Community - Who are We?}
    The is a group of people established primarily in Mauritius
    who are
    
    \begin{itemize}
    \item Passionate Makers
    \item Hackers
    \item Enthusiasts
    \item Hardware Developers 
    \item Inventors
    \item Designers
    \item Tinkerers
    \item Craftman
    \end{itemize}
    
  \end{block}

}
    
\frame{
  \frametitle{An introduction to the The Mauritius Makers Community -
    Part 2}
  
  \begin{block}{The Mauritius Makers Community - What we are
      passionate about}
    
    geared primarily toward technological innovation such as
    
    \begin{itemize}
    \item Electronics
    \item Open Hardware
    \item Internet Of Things
    \item Robotics
    \item Small Board Computers
    \item Microcontrollers
    \item Embedded Software
    \item Printed Circuit Board Design
    \item DIY - Do It Yourself
    \item DIWO - Do It With Others
    \item CAD - Computer Aided Designing
    \item Wearables
    \item 3D Printing
    \item Plastic, Wood \& Metal Work
    \end{itemize}

  \end{block}

}

\frame{
  \frametitle{Pritvi Jheengut}
  
  \begin{block}{Who am I?}
    \begin{itemize}
    \item Name : Pritvi Jheengut 
    \item Empl : Meteorological Services 
    \item Post : Senior Meteorological Telecommunication
      Technician
    \item Work : Maintain and repair Linux Workstations
      and Automatic Weather Stations
    \item Else : Co-founder of Mauritius Makers Community during
      Jochen's Keynote and Introduction Of The MSCC at the Developers
      Conference 2015
    \item Else : Vice-President of The Linux User Group Meta
    \item Else : Craftman At MSCC
    \item Want : Create the Mauritius Local Guide
    \item Want : Corsairs Hackers Reboot - October 2018 
    \end{itemize}
  \end{block}

}

\subsection{Communications}

\frame{
  \frametitle{What is Communication!}

  \begin{block}{Communication theory}
    Communication involves two parties, one a sender, the second one
    a receiver.

  \end{block}

  \pause
  
  \begin{alertblock}{Why we need Communication?}
    WHY???
  \end{alertblock}

}

\subsection{Communication Protocols}

\frame{
  \frametitle{Communication Protocols used by Makers}

  \begin{block}{Some Communication Protocols used by Makers}

    \begin{itemize}
    \item SPI
    \item I$^2$C
    \item CAN
    \item SMBus
    \end{itemize}
  \end{block}

}

\frame{
  \frametitle{SPI}

  \begin{block}{Apropos SPI - Serial Peripheral Interface}
    \begin{itemize}
    \item  SPI, Serial Peripheral Interface is a single master,
    multi-slaves four wire variable speed synchronous
    message serial protocol.
    \item 
    It was originally developed by Motorola in the 1980's
    and has become a de facto standard
    \item SPI is  widely used by
    microcontrollers to talk with sensors, eeprom and flash
    memory, codecs and various other controller chips, ADC
    \& DAC converters, and more.
    \end{itemize}
  \end{block}

}

\frame{
  \frametitle{I$^2$C}

  \begin{block}{Apropos I$^2$C - Inter-Integrated Circuit}
    \begin{itemize}
    \item     I$^2$C, Inter-Integrated Circuit is a multi-master,
    multi-slave two-wire variable speed synchronous packet
    switched serial protocol used in many microcontroller
    applications. 
    \item  It was originally developed by NXP, Philips in the 1980's
    and  provides an inexpensive bus for connecting many
    types of devices  with infrequent or low bandwidth
    communications needs.
    \item  I$^2$C is widely implemented in embedded systems.
    \item  Since October 10, 2006, no licensing fees are required
    to implement the I$^2$C protocol. However, fees are
    required to obtain I$^2$C slave addresses allocated by NXP.\\
    Source :: wikipedia
    \end{itemize}
    
  \end{block}

}

\frame{
  \frametitle{CAN}

  \begin{block}{Apropos CAN - Controller Area Network}
     \begin{itemize}
    \item CAN, Controller Area Network is a multi-master, two or more
    wires variable speed message based serial protocol to connect
    two or more nodes.
    \item   It was originally developed by Bosch which
    has widespread use in automation, embedded devices,
    marine, industrial, medical, automotive as well as aeronautical
    fields. 
    \item  Communication can be allowed over a USB or Ethernet port.\\
    With one common cable and implemented on both hardware and
    software, the CAN protocol enables several piece of electronic
    equipment to be connected to each other.
    \end{itemize}
    
  \end{block}

}

\frame{
  \frametitle{SMBus}
  
  \begin{block}{Apropos SMBus - System Management Bus}
     \begin{itemize}
    \item  SMBus, System Management Bus is a multi-master,
    multi-slave two-wire variable speed synchronous packet
    switched serial protocol used in many microcontroller
    applications. 
    \item  It is a subset of I$^2$C and heavily used in many
    Computer Motherboards especially those having an Intel
    Chipset for reading sensor values such as temperature,
    voltage, fan speed,...
    \item  Modern I$^2$C is compatible  with SMBus.
    \end{itemize} 
  \end{block}
  
  \begin{block}{Apropos PMBus - Power Management Bus}
    A special mention : PMBus, Power Management Bus is a
    variant of SMBus targeting power supplies. 
  \end{block}
  
}

\section{Communication with gadgets using UART}

\frame{
  \frametitle{UART}

  \begin{block}{Apropos UART - universal asynchronous
      receiver-transmitter}
    UART, Universal Asynchronous Receiver Transmitter
    is a hardware device for asynchronous serial
    communication in which the data format and
    transmission speeds are configurable. 
\end{block}

}

\end{document}
